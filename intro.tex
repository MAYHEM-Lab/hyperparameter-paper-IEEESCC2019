Serverless~\cite{ref:serverless} computing is a rising function-based (also known as Funtions-as-a-Service~\cite{ref:faas}) programming and deployment paradigm on top of cloud infrastructure, in which programmers manifest business logic without concerning provisioning servers by modular functions that are triggered by incoming events and return computation results in serialized format like XML or JSON. Such events include HTTP requests from clients, possibly proxied by API Gateway, or communications between cloud services based on heterogeneous protocols. Depending on the supported runtimes from cloud vendors, functions could be written in various high-level languages (i.e. NodeJS, Python, Java, etc.) and take advantage of serverless SDK to interact with other cloud services. Until recently, the leading serverless platform, Amazon Web Services(AWS) Lambda, expands language support by custom runtime~\cite{ref:customruntime} where software engineers can implement functions by any programming languages. 

Primarily, there are four evolutionary advantages that serverless architecture brings to cloud computing. First and foremost, it raises the abstraction layer from virtual machines to function containers~\cite{ref:container} that abstracts away the responsibilities of infrastructure management from DevOps~\cite{ref:devops} personnel to the cloud provider, in terms of scaling, patching, provisioning, error-handling and terminating underlying bare metal servers or clusters. Such shift allows more agile development life cycle and more product-oriented programming iteration. Secondly, the event-driven and paralleled container runtime of serverless function provides more fine-grained computational resource isolation and usage. Hence, serverless architecture makes cloud operation more secure and energy-efficient compared to traditional virtual machine instance runtime. Thirdly, serverless architecture decomposes monolithic application code base even further from microservices~\cite{ref:microservices} into granular functions, which are easier to be created, maintained and refactored by team or individual programmer. Furthermore, Function-as-a-Service applications could be autoscaled independently with respect to modular functions depending on the volume of request. This capability also contributes to the energy efficiency and eliminating throttling bottlenecks. Lastly, serverless computing provides "Pay-per-execution" pricing model~\cite{ref:pricing}. In a real world case study~\cite{ref:serverless_econ}, the server operational cost on serverless platform could be reduced dramatically, given a relatively lower amount of Transactions Per Second (TPS). Based on these four advantages, serverless architecture appropriately serves mid-size stateless application under certain TPS threshold better than virtual machine based clusters. According to these merits, the stateless, event-driven and embarrassingly parallel tasks are potentially cost-effective with lower latency by running on serverless framework. 

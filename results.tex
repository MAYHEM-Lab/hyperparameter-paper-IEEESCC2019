\begin{table}[t]
\centering
%\scriptsize
%\resizebox{\columnwidth}{!}{
\begin{tabular}{|l|l|} \hline
\textbf{Application}& \textbf{Description}\\
\hline
Prophet& Time series decomposition and prediction\\ 
\hline
Multi-Regression& Multiple linear regression/prediction of time series\\
\hline
XGBoost& Regression and classification by gradient boosting\\
\hline
SVC & Classification based on support vector machine\\
\hline
NN& Classification by layered artificial neural network\\
\hline
\end{tabular}
%}
\caption{Machine learning applications used as benchmarks
to evaluate Seneca. 
%paper submission is blind (so such references must be omitted so as not to reveal our identities):
%Code base is available at project repository~\cite{ref:seneca}.
\label{tab:bmarks}}
\vspace{-0.2in}
\end{table}

In this section,
we evaluate Seneca in terms of performance, monetary cost, 
and hyperparameter tuning time (latency), 
for different machine learning applications.
We first overview these applications and our empirical methodology. 
We then present our experimental results. 

\begin{table}[t]
\centering
\scriptsize
%\resizebox{\columnwidth}{!}{
\begin{tabular}{|l|l|l|} 
\hline
\textbf{Hyperparameter}& \textbf{Default} & \textbf{Tuning options}\\
\hline
grow & linear & [linear, logistic] \\
\hline
changepoint prior scale & 0.05 & [0.05, 0.5] \\
\hline
holidays prior scale & 10 & [1, 5, 10] \\
\hline
seasonality prior scale & 0.5 & [0.1, 0.5] \\
\hline
fourier order & 10 & [5, 10, 15, 20] \\
\hline
seasonality mode & additive & [additive, multiplicative] \\
\hline
interval width & 0.8 & [0.5, 0.8] \\
\hline
\end{tabular}
%}
\caption{The hyperparameters that Seneca considers for Prophet. 
The default value and tuning options are listed. 
\label{tab:prophet_para}}
\vspace{-0.2in}
\end{table}

\begin{table}[t]
\centering
\scriptsize

\begin{tabular}{|l|l|l|} 
\hline
\textbf{Hyperparameter}& \textbf{Default} & \textbf{Tuning options}\\
\hline
%hidden layer size & 100 & [100] \\
%\hline
max depth & 3 & [2,3,4]\\
\hline
learn rate & 0.1 & [0.1,0.01] \\
\hline
N est & 100 & [100,300,400] \\
\hline
obj & reg:linear & [reg:linear,rank:pair] \\
\hline
booster &gbtree& [gbtree,gblinear,dart] \\
\hline
n jobs & 1& [1,4]\\
\hline
gamma & 0 & [0,0.01] \\
\hline
min childwt & 1& [0.1,1,2] \\
\hline
scale poswt & 1& [0.1,1,2] \\
\hline
samp tree & 1& [0.2,0.5,1]\\
\hline
samp level & 1& [0.2,0.5,1]\\
\hline
ralpha & 0& [0.0.1,0.9] \\
\hline
rlambda & 1& [0.1,0.9,1] \\
\hline
base score & 0.5& [0.5,1,10] \\
\hline
%random state & 123& [123] \\
%\hline
\end{tabular}

\caption{The hyperparameters that Seneca considers for XGBoost. 
The default value and tuning options are listed. 
\label{tab:xgboost_para}}
\vspace{-0.2in}
\end{table}

\begin{table}[t]
\centering
\scriptsize

\begin{tabular}{|l|l|l|} 
\hline
\textbf{Hyperparameter}& \textbf{Default} & \textbf{Tuning options}\\
\hline
C & 1.0 & [0.5, 1.0] \\
\hline
kernel & rbf & [rbf, linear, poly, sigmoid] \\
\hline
degree & 3 & [3, 4] \\
\hline
gamma & auto& [auto, scale] \\
\hline
coef0 init & 0.0 & [0.0, 1.0] \\
\hline
%shrink & True & [True] \\
%\hline
probability & False & [False, True] \\
\hline
tol & 1e-3& [1e-3, 1e-4] \\
\hline
%cache & 5.0& [5.0] \\
%\hline
%max iter & 20 & 20 \\
%\hline
decision function shape & ovr & [ovo, ovr] \\
\hline
%random state & 123& [123] \\
%\hline
\end{tabular}

\caption{The hyperparameters that Seneca considers for SVC. 
The default value and tuning options are listed. 
\label{tab:svc_para}}
\vspace{-0.2in}
\end{table}

\begin{table}[t]
\centering
\scriptsize

\begin{tabular}{|l|l|l|} 
\hline
\textbf{Hyperparameter}& \textbf{Default} & \textbf{Tuning options}\\
\hline
%hidden layer size & 100 & [100] \\
%\hline
activation & relu & [id,tanh,relu] \\
\hline
solver & adam & [lbfgs,sgd,adam] \\
%\hline
%alpha & 0.0001 & [0.0001] \\
\hline
batch size & auto & [200,auto] \\
\hline
rate & const& [const,invsc,adapt ] \\
\hline
rate init & 0.001 & [0.001,0.0001] \\
\hline
power T & 0.05 & [0.1,0.5,2] \\
%\hline
%max iter & 20& [20] \\
%\hline
%shuffle & True& [True] \\
%\hline
%random & 123& [123] \\
\hline
tol & 1-e4& [1e-3,1-e4,1-e5] \\
%\hline
%momentum & 0.9& [0.9] \\
%\hline
%early stop & False& [False] \\
%\hline
%beta1 & 0.9& [0.9] \\
%\hline
%beta2 & 0.999& [0.999] \\
%\hline
%epsilon & 1e-8& [1e-8] \\
\hline
iter nchange & 10& [5,10,20] \\
\hline
\end{tabular}

\caption{The hyperparameters that Seneca considers for NN.
The default value and tuning options are listed. 
\label{tab:nn_para}}
\vspace{-0.2in}
\end{table}


\subsection{Benchmark Applications and Training/Testing  Datasets}
The machine learning applications that we use to evaluate Seneca 
are shown in Table~\ref{tab:bmarks}, each with 
a brief description.
Prophet~\cite{ref:prophet} is a time series analysis library built and maintained by Facebook and the open-source community. The input dataset is a time series of view counts
%logorithm
of Payton Manning's wikipedia page from Dec. 10th, 2007 to Jan. 20th, 2016. 
The dataset exhibits both seasonality and a holiday effect (e.g. around the super bowl games). 
We use the first 6 years as the training set and the last 2 years as the testing set.
We use a cross-validation horizon (sliding window) of 1-year, 
and a period (sliding pace) 
of 180 days.  As such, Seneca performs three cross-validations for each 2-year testing range.

Prophet expects multiple hyperparameters: \textit{growth} specifies linear or logistic trend model growth; \textit{prior scale} indicates the strength of the 
sparse prior probability~\cite{ref:sparse_prior}. 
There are three prior scale hyperparameters for change point, holidays, and seasonality. 
Since Prophet uses a Fourier sum to estimate seasonality, 
the \textit{fourier order} is the number of terms in the partial Fourier 
sum. \textit{Seasonality mode} indicates that the effect of seasonality is either 
multiplicative or additive. Finally, the width of uncertainty intervals 
is set using \textit{interval width}.

In the results that follow, we evaluate \textit{default} parameteterizations (that come with 
the application or that are defined by the application developer) for each application.  
The default parameters and tuning options for Prophet are listed in the table~\ref{tab:prophet_para}. For every cross validation, the application computes mean square error (MSE) as $\frac{1}{n}\sum_{i=1}^{n}(Y_i - \hat{Y_i})$, where $\hat{Y_i}$ is the ground truth, $Y_i$ is model prediction and n is the number of data points. Then the application returns the average MSE of all cross validation. Based on the model trained by default parameter setting, this average MSE metric is 0.284.


Multi-Regression is a regression application 
from an IoT project~\cite{iot-cpu} that was developed by the authors
to predict outdoor temperature from the processor 
temperature of single board computers (SBCs).  
The technique uses multiple linear regression models computed from time
series of processor temperature measurements,
to predict outdoor microclimate temperatures without using a temperature sensor.
%(e.g. for use in 
%precision agriculture applications such as irrigation control and frost protection).
The training dataset consists of eight input time series (one per SBC, each containing 
5-minute measurements) from April 5th, 2018 to Dec. 10th, 2018.

A hyperparameter configuration for this application is a subset of input SBC time series.
Seneca considers all \texttt{$2^N - 1$} potential subsets (for $N$ input time series).
For this application,
we assume the default parameterization is the full set of input time series (8 in this case).
The test dataset for this application is a time series of the outdoor temperature (ground truth) 
over the same period.  The application makes predictions for each of these outdoor temperatures
using the regression coefficients constructed from the training set
and computes the mean square error between predicted and ground truth value pairs.


XGBoost~\cite{ref:xgboost-web} is an open source framework for gradient boosting, which 
performs both regression and classification. We consider it a regression application and
a classification application in this evaluation.  
The hyperparmeters and their defaults that Seneca uses in this study are listed in Table~\ref{tab:xgboost_para}. A definition for each hyperparameter can be found at~\cite{ref:xgboostparams}.
%xgboostparams: https://xgboost.readthedocs.io/en/latest/python/python_api.html#module-xgboost.sklearn
Support vector classification (SVC)~\cite{ref:svc} is a classification algorithm 
based on support vector machine and implemented by libsvm~\cite{ref:libsvm} library.
The hyperparmeters and their defaults that Seneca uses for SVC in this study are listed in Table~\ref{tab:svc_para}. A definition for each hyperparameter can be found at~\cite{ref:svcparams}.
%xvcparams: https://scikit-learn.org/stable/modules/generated/sklearn.svm.SVC.html
Neural network (NN)~\cite{ref:neural_network} is a machine learning framework that learns 
and make decisions from patterns in an input dataset. For this application, 
we use a feedforward multilayer perceptron model~\cite{ref:feedforward_nn} 
for the classification task.
The hyperparmeters and their defaults that Seneca uses for Neural Network in this study are listed in Table~\ref{tab:nn_para}. A definition for each hyperparameter can be found at~\cite{ref:nnparams}.
%nn: https://scikit-learn.org/stable/modules/generated/sklearn.neural_network.MLPClassifier.html

For these applications (XGBoost, SVC, and NN), we use a labeled
dataset for training, testing, and evaluation from this
project~\cite{iot-cpu}. The dataset contains measurements from individual
citrus fruit (e.g. oranges, mandarins, lemons, etc.) taken by a fruit sorting
and grading device (using a large number of sensors).  The measurements (i.e.
features) include size, shape, weight, color, diameter, flatness, among other
characteristics, for each fruit.  The label indicates which field the fruit
was harvested from.  The applications train a model on a subset of the
data.  They then use this model to predict the field
from which each fruit originates. For this study we use the same random 80\%
subset of the dataset as the training dataset.  We use the remaining 20\% as the
testing dataset.

These applications compute classification accuracy score
as $\frac{1}{n}\sum_{i=1}^{n}1(Y_i = \hat{Y_i})$, where $Y_i$ is the
prediction class, $Y_i$ is the true class, $n$ is the number of samples, and 1(x)
is the indicator function. XGBoost also computes MSE for its regression task
on the same datasets. The dataset has been filtered to remove correlated
features (those with an absolute value of the Pearson correlation coefficient
greater than 0.8). The dataset has also been balanced by downsampling one
target whose number greatly exceeds other ones. The resulting dataset contains
33926 rows (individual fruit) distributed evenly to 5 targets and each row has
18 features.

%The dataset is from a Compac packline machine~\cite{ref:compac} that sorts and grades fruit using a large number of 
%measurements (i.e. feature measurements including size, shape, weight, color, diameter, flatness, etc.).  The labels identify the field from which fruit was harvested.
%The dataset is filtered to remove correlated features (those with an absolute value of the Pearson correlation 
%coefficient greater than 0.8).  

%Seneca downloads the dataset from AWS S3 for each function invocation.
%It then randomly indexes the dataset and splits it into two.  

% For these applications (XGBoost, SVC, and NN), we consider three
% split scenarios for training and testing: 80\%-20\%, 20\%-80\% and
% 0.1\%-99.9\%. As the proportion of training data decreases, the performance of
% classifier degrades accordingly. However we note that using the optimized
% hyperparameters generated by Seneca, these classifiers are able to achieve
% accuracy levels similar to those generated from the default parameters, but
% with significantly less training data.

% and we argue that Seneca is able to eke out
%improvement even with very limited training data.

%\textcolor{blue}{list them here once we finish the additional results} 
%This setup also means that each Seneca function
%considers a different training and testing set during hyperparameter tuning. 

%todo in the future -- for a complete tuning, use the same testing and training sets
%run this multiple times (for the same test/train datasets) and report the average and stdev in the latency (and memory?) results.
%todo in the future -- repeat the above using different test/train datasets, but keep 
%them the same for all functions in a single tuning run.  Compare the MSE/accuracy and
%the configurations that are selected, analyze how they change (if they do)

\subsection{Empirical Methodology}

To evaluate Seneca, we measure tuning performance, application execution time,
and monetary cost.  For tuning performance, we use mean squared error (MSE)
for regression tasks and percentage accuracy for classification tasks (i.e. we score
model prediction accuracy and not explanatory power).  Because we use the name 
total number of hyperparameters for each model, the penalty function in terms of 
Bayesian Information Criteria, is the same. Thus, we can compare models using
this score.

The default configuration that we consider for each application is described
above.  We report results for the default, best (Seneca's recommendation), and
worst performing configurations.  Seneca computes all possible combinations of
the hyperparameter settings specified in the configuration to extract each of
these results.  The default results represents those that a novice or first
time user might employ when using these applications as a ``black box.''  The
worst performing scenario shows how bad the results can be when parameters are
poorly tuned.  Finally, the best is the upper bound on what is possible from
tuning the hyperparameters for the values and dataset specified 
(e.g. using expert knowledge or Seneca). 

To evaluate the cost savings introduced by Seneca, we deploy the applications
on Amazon Web Services (AWS) Lambda~\cite{ref:awslambda} and extract
execution time and memory use from AWS CloudWatch~\cite{ref:awscloudwatch} logs.  
We then compute monetary cost
using the AWS Lambda pricing model~\cite{ref:pricing}.
%awslambdapricing: https://aws.amazon.com/lambda/pricing/
Each function downloads the training and testing dataset 
of the application from AWS Simple Storage Service (S3)~\cite{ref:awss3} upon function invocation. 
We do not consider the cost of dataset storage in our cost computations (it is very small).

%We also experiment with different memory configurations for Seneca functions
%in AWS Lambda.  
We also investigate Seneca's automatic memory optimization capabilities.  To
do so we compare the execution performance and cost of the benchmarks using
the maximum allocatable memory size to the performance and cost when run with
Seneca's automatically determined memory size.  Note that this comparison
assume that the memory size used by each benchmark does not depend on
hyperparameter settings.  We have verified that this assumption holds for the
benchmarks and datasets we consider and plan to consider applications for
which hyperparameter settings require different maximum amounts of memory as
part of future work.  

\subsection{Application Efficacy}

We begin by evaluating Seneca's effect on application efficacy.
That is, we detail the effect on the quality of the output generated by each
application when Seneca determines the hyperparameter settings. 

\begin{table}
\centering
\begin{tabular}{|c|c|c|c|}
\hline
& Prophet & Multi-Regression & XGBoost\\
\hline
\# of Combinations & 384 & 255 & 768\\
\hline
\hline
Default MSE & 0.284 & 11.446 & 0.119 \\
\hline
Worst MSE & 1.266 & 43.752 & 8.98 \\
\hline
Best MSE (Seneca) & 0.220 & 9.621 & 0.044 \\
\hline
%percent difference between default and best 26.07%, 15.94%, 67.30%
%percent difference between worst and best 82.63%, 78.01%, 95.52%
%\begin{figure}[t] \centering 
%\includegraphics[scale=0.4]{mse}
\end{tabular}
\caption{Combination count and MSE for the default, best (Seneca's recommendation), and worst hyperparameter configurations for the three regression applications. 
For the MSE  values (rows 3-5), lower is better.
\label{tab:mse}}
\vspace{-0.2in}
\end{table}

\begin{figure}[t] \centering 
\vspace{-0.2in}
\includegraphics[scale=0.36]{box_plot_mse_2}
\vspace{-0.4in}
\caption{The box plot of MSE metric from three regression applications in the entire hyperparameter tuning search space. The red notch shows the MSE from the default settings. The colored diamonds are outliers beyond two interquartile ranges. Lower MSE values are better.
\label{fig:box_plot_mse}}
\vspace{-0.2in}
\end{figure}

%using the three regression applications: 
%Prophet, Multi-Regression, and XGBoost (regression task). 
Table~\ref{tab:mse} shows the effect of Seneca on the quality of the
regressions generated by Prophet, Multi-regression, and XGBoost (when used as
a regression algorithm). The first row indicates the number of
different hyperparameter settings Senica considered.  Rows 3 through 5 show
the MSE generated by the default hyperparameter settings, the best (lowest
MSE) Seneca was able to find, and the worst (largest MSE) that Seneca tested
for each regression algorithm and dataset.

%s mentioned previously, the table shows regression
%quality in terms of MSE. 
%
%size of the hyperparameter search space (number of configurations
%considered) and the mean squared error (MSE)
%for each of the regression applications that we consider (columns 3-5).
%The first row of data is 
% the number of hyperparameter configurations in the search space of each. 
%The last three rows show the MSE for the default, worst, and best performing 
%hyperparameter configuation reported by Seneca (lower is better).  

%Note that Seneca's recommendation is the best performing configuration.
Compared to the default hyperparameter settings,
Seneca reduces MSE by 26\%, 16\%, and 63\%, for Prophet, Multi-Regression, and XGBoost,
respectively, for the datasets and training methodologies that we consider.
Verses the worst case, Seneca reduces MSE by 83\%, 78\%, and 99\%, respectively.

Figure~\ref{fig:box_plot_mse} provides a box plot of MSE (lower is better) for
these applications for the entire hyperparameter search space. The central
rectangle covers the interquartile range (IQR), which is defined as the range
of data points from first quartile to third quartile (Q3 - Q1).  The upper
whisker extends to the last datum less than \texttt{$(Q3 + 2 * IQR)$} and the
lower whisker extends to the first datum greater than \texttt{$(Q1 - 2 *
IQR)$}. The data points beyond the whiskers are considered outliers and are
plotted as colored diamonds. The red notch identifies the MSE that results
from training the model using the default settings of hyperparameter. The
difference between red notch of lowest data point is improvement brought about
through the use of Seneca. From the location of red notch, we can see the MSE
metrics from default configuration in three applications are close to the
first quartile value in all three cases (slightly greater that Q1 for Prophet
and Multi-regression and slight less for XGBoost).  However Seneca's MSE
performance is congruent with the end of the lower whisker.  Overall, however,
the quality of the application's output (for a given dataset) is dramatically
affected by hyperparameter settings.  Perdictably the default settings are
near the ``good'' end of the spectrum, but as both Table~\ref{tab:mse} and
Figure~\ref{fig:box_plot_mse} indicate, Seneca is able to find
parameterization that improves output quality over the default settings
in each case.
%Thus for Prophet
%and Multi-Regression, Seneca 
%, meaning the potential room for improvement is about 75\%
%over the range of MSE metric. The box plot also illustrates that the best MSE
%of Prophet and multivariate regression are among outliers, where a
%comprehensive search is very critical to find the best configurations. 
%\textcolor{blue}{Add more analysis here to discuss the variance in the MSE
%for the range of configurations and how they change across applications.
%Include any insights that the boxplots add.}

%%%%%%%%%%%%%%%%%%%%% classification apps - tuning performance
%%%%%%%%%%%%%%%%%%%%%%%%%%
We next empirically evaluate the tuning performance of Seneca for the three
classification applications: XGBoost (classification), SVC, and NN.
Table~\ref{tab:accuracy} presents the accuracy percentage (higher is
better) reported by each application (3 right-most columns) for the default,
worst, and best hyperparameter tuning configurations reported by Seneca.
using an 80/20 (train/test) percent split,
Seneca increases accuracy by 3.43\%, 62.52\%, and 0.53\%, for XGBoost
(classification), SVC, and NN applications, respectively.  Verses
the worst case, Seneca improves accuracy by 99.13\%, 75.75\%, and 65.57\%,
respectively.


\begin{table}
\centering
\begin{tabular}{|l|c|c|c|}
\hline
\textbf{80\%-20\%} & XGBoost & SVC & NN\\
\hline
\hline
Default Accuracy & 95.70\% & 21.77\% & 84.23\%\\
\hline
Worst Accuracy&0.00\%&7.53\% & 19.19\%\\
\hline
Best Accuracy (Seneca) &99.13\% & 83.29\% & 84.76\%\\
\hline
%\hline
%\hline
%\textbf{20\%-80\%} & XGBoost & SVC & NN\\
%\hline
%Default Accuracy & 95.07\% & 21.31\% & 55.44\%\\
%\hline
%Worst Accuracy&0.00\% & 9.42\% & 19.55\%\\
%\hline
%Best Accuracy (Seneca) & 98.68\% & 43.61\% & 60.89\%\\
%\hline
%\hline
%\textbf{0.1\%-99.9\%} & XGBoost & SVC & NN\\
%\hline
%Default Accuracy & 86.37\% & 20.02\% & 20.00\%\\
%\hline
%Worst Accuracy & 0.00\% & 19.99\% & 19.79\%\\
%\hline
%Best Accuracy (Seneca) & 89.14\% & 55.79\% & 29.99\%\\
%\hline

%percent difference between default and best 83.65%, 83.84%, 79.83%
%percent difference between worst and best 86.07%, 85.73%, 84.63%
%\begin{figure}[t] \centering 
%\includegraphics[scale=0.38]{accuracy}
\end{tabular}
\caption{The Accuracy reported for the 
default, best (Seneca's recommendation), and worst hyperparameter configurations for 
the three classification applications using 80\% of the data to train and 20\%
of the data as a test set. 
%under three train/test split scenarios.  
For the accuracy values in the table, higher is better.
\label{tab:accuracy}}
\vspace{-0.2in}
%\end{figure}
\end{table}


The 80-20 case represents the typical usage of these classification
applications.  Often, when evaluating the accuracy of a specific classifier,
the user uses the 80-20 split ``rule'' when measuring accuracy.  In this case,
Seneca's improvement over the default parameterizations varies from less than
1\% for the Neural Network (NN) application (column 4), to 61.5\% for SVC
(best versus default in column 3).

\begin{figure}[t] \centering 
\includegraphics[scale=0.36]{box_plot_Accuracy_20}
\vspace{-0.4in}
\caption{The box plot of accuracy metric from three classification applications for the entire hyperparameter tuning search space. The red notch indicates the accuracy metric from the trained model using default settings. The colorful diamonds are outliers beyond two interquartile ranges. Higher accuracy values are better.
\label{fig:box_plot_accuracy}}
\vspace{-0.1in}
\end{figure}

Figure~\ref{fig:box_plot_accuracy} presents the accuracy (higher is better) 
for the entire hyperparameter search space for each classification
application.  The central rectangle covers from first quartile to third quartile (Q3 - Q1) and the whiskers span from \texttt{$(Q3 + 2 * IQR)$} to \texttt{$(Q1 - 2 * IQR)$}. The red notch indicates the accuracy metric from the model trained by default settings and colored diamonds are outliers beyond whiskers.
For both types of applications (regression and classification), Seneca significantly
improves tuning performance over the default and worst case configurations for 
XGBoost and SVC while generating only a small improvement over default for NN.


\subsection{Cost Analysis}

We next analyze the monetary cost effects introduced by Seneca.  To do so, we
use Seneca to find the hyperparameterization that optimizes application output
quality (lowest MSE for regression, highest accuracy for classification).
However note that trying all possible hyperparameterizations incurs a cost.
Further, because Seneca uses AWS Lambda, the amount of memory it uses during
tuning affects this cost. 


Thus we compare two Seneca execution configurations.  In the first
we configure each lambda function to
use the maximum amount of memory allowed by AWS Lambda (3008 MB) for all
executions.
Second, we
employ Seneca's memory optimization which automatically detects and configures
the amount of memory needed by each function.  This optimization requires
Seneca to ``probe'' to determine the best memory size to use.  The
cost of these probes
(denoted ``Optimizer Cost'') must be taken intio account as well.
%Because memory impacts both
%cost and execution time (total latency of Seneca hyperparameter tuning), we
%present empirically evaluate both in this subsection.  
Note that XGBoost's performance profile is identical when used for reghression
or classification so we include its measurements only once in the following
analysis.
%For each, we only
%include XGBoost once since its performance profile is the same regardless of
%whether we are using regression or classification.

\begin{table}
\centering
\begin{tabular}{|l|c|c|c|c|c|}
\hline
& Prophet & MR & XGBoost & SVC & NN\\
\hline
\hline
Exec time (minutes)& 24.96 & 4.18 & 17.48 & 0.91 & 2.49 \\
\hline
Seneca Cost (\$) & 0.722 & 0.121 & 0.5 & 0.023 & 0.072 \\
\hline
\end{tabular}
\caption{Monetary cost of using Seneca for hyperparameter tuning when configured to use the maximum amount of memory (3008 MB) possible for each function invocation. 
\label{tab:cost_max_memory}}
\vspace{-0.2in}
\end{table}

\begin{table}
\centering
\begin{tabular}{|l|c|c|c|c|c|}
\hline
& Prophet & MR & XGBoost & SVC & NN\\
\hline
\hline
Exec time (minutes)& 38.7 & 8.72 & 29.07 & 0.91 & 3.17 \\
\hline
Optimizer Cost (\$) &0.033 & 0.014 & 0.003 & 0.010 & 0.008 \\
\hline
Seneca Cost (\$) &0.62 & 0.086 & 0.354 & 0.007 & 0.056 \\
\hline
Total Cost (\$) &0.653 & 0.1 & 0.357 & 0.016 & 0.064 \\
\hline
\hline
Savings (\$) & 0.07 & \$0.02 & 0.15 & 0.007 & 0.008\\
\hline
Savings (\%) & 10\% & 17\% & 30\% & 30\% & 11\%\\
\hline
\end{tabular}
\caption{Seneca Memory Optimization:  The  monetary  cost  of using Seneca for hyperparameter tuning when the amount of memory  used  for  each  function  invocation  is  automatically inferred and set by Seneca.
\label{tab:cost_optimized}}
\vspace{-0.2in}
\end{table}

Tables~\ref{tab:cost_max_memory} and~\ref{tab:cost_optimized} show the results with and without 
the Seneca memory optimization, respectively. In both tables, the five columns identify the 
benchmark applications and the first row is the total execution time in minutes for 
Seneca's hyperparameter tuning for the corresponding application.  
In Table~\ref{tab:cost_max_memory}, the second row is total cost of tuning when Seneca's 
lambda functions are configured to use the maximum amount of memory (3008 MB). 
Table~\ref{tab:cost_optimized} shows the cost of the memory optimizer in the
second row, Seneca's cost using the optimization in the third row, and total cost on fourth row.
Finally, in the fifth and six rows of Table~\ref{tab:cost_optimized} we show the
savings that memory optimization introduces over using the maximum memory in
dollar cost and as a percentage respectively.   

These tables illustrate two important cost features of Seneca.  First, using
AWS Lambda, full hyperparameter space exploration is not expensive in 
\textit{absolute} dollar cost terms.  Even using the maximum memory available
for almost 25 minutes of execution time (cf Prophet in Table~\ref{tab:cost_max_memory}), the
total dollar cost is approximately 73 cents.  Secondly, Seneca's memory
optimizations produce a significant percentage savings which, over repeated
executions, is important.  

Note also that Seneca's memory/cost optimization
introduces at an execution time penalty (relative to the maximum memory
execution) for some of the applications.  That is, the
execution times for all applications except SVC increase when Seneca chooses
a memory size.  Thus while Seneca is finding the best application output
quality and optimizing the cost of producing that output, it is doing so at
the expense of execution performance (time to solution). Such penalty is expected due to the proportional CPU power to allocated memory. However, Seneca optimizer is able to find the optimal memory configuration to minimize such penalty. In the case of SVC, Seneca saves 30\% compared to max memory execution (\$0.016 vs \$0.023) but has the same execution time (0.91 minutes).
% The source of this execution performance penalty is not clear.  We hypothesize
% that when requesting a large memory for AWS Lambda, Amazon chooses a machine
% that also has fast processors so the Lambda functions execute more quickly.
% If this supposition is true, then the AWS Lambda pricing model correctly
% weights the cost of additional memory (in light of the reduced execution time)
% because the large-memory executions are still more expensive.
%However regardless of the cause, 
Even in the worst case of CPU-bound application, the Seneca memory optimizer still provides a way to trade off time to solution for cost when the highest possible application output quality is desired.
% \textcolor{blue}{AWS announces the proportional CPU power to memory which is the source of penalty}
% \textcolor{blue}{See my notes to fix the tables in the caption of these figures.}



% \textcolor{blue}{It would be good here in the Seneca case to also run the applications in the minumum amount of memory discovered by Seneca (ie if we were to run Seneca ahead of time, taking the max memory used by any of the functions per application) -- to show what Seneca is missing.  this assumes you are doing the memory optimization WHILE you are doing the tuning -- if you are not, you can leave this out, but we need to remind the reader at this point how the memory optimization is done by Seneca (we should do this in this paragraph, either way).}

% \textcolor{blue}{Comments: The minimum amount of memory is the "max memory used" value. I tried that out and the execution time increases dramatically at the minimum memory case. I will explore the idea of simultaneous optimization and tuning later}




%\begin{figure}[t] \centering 
%\includegraphics[scale=0.36]{Cost_analysis_2}
%\caption{The cost analysis of Seneca automatics memory optimization. 
%By reducing the memory use of each, Seneca reduces the monetary cost of hyperparameter tuning by 
%an average of 20\% over using the maximum memory allocatable.\label{fig:cost_analysis}}
%\vspace{-0.2in}
%\end{figure}

%\begin{figure}[t] \centering 
%\includegraphics[scale=0.36]{Latency_analysis_2}
%\caption{Latency of Seneca with (Opt) and without (No Opt) memory optimization, for each application. As Seneca reduces memory use (and cost), it also extends the execution time of the functions (and thus total tuning time, i.e., the latency of Seneca) by Y\% on average across applications \textcolor{blue}{compute and fill in Y here.}
%\textcolor{blue}{Change the legend as follows: change "Optimized Memory" to "Opt"; change "Maximum Memory" to "No Opt"; Make sure that Opt stays green and change No Opt to be red (same as above graph)} 
%\label{fig:latency_analysis}}
%\vspace{-0.2in}
%\end{figure}

%Figure~\ref{fig:cost_analysis} presents the breakdown in the monetary cost of
%using Seneca (performing hyperparameter tuning) with and without its automatic
%memory optimization.  We plot bar graphs for each application across the
%x-axis; the y-axis is monetary cost in US dollars. The stacked column is the
%total cost of tuning by Seneca Memory Optimization, and the green and red
%segments represent the cost of tuning and function optimization respectively.
%The red column at the right-hand side of each application is the total cost of
%tuning by using lambda function at maximum memory. For computation intensive
%applications like Prophet, they generate higher costs, which depend on the
%duration and memory usage on AWS Lambda. %The difference between two costs 


%\textcolor{red}{Add analysis here on what the columns are showing -- discuss
%any anomalies (ie why is prophet optimization cost more than the others; why
%mult-regression difference is greater than the others). provide a summary for
%that states the average difference between each pair of bars across
%applications.}

%By reducing the amount of memory used, Seneca reduces its cost.  However by
%doing so, Seneca also increases the total time it requires to perform
%hyperparameter tuning (latency) as shown in Figure~\ref{fig:latency_analysis}
%(due to the memory constraints place on execution).  The x-axis shows two bars
%per application, one with memory optimization (Opt) and one without (No Opt).
%The y-axis is Seneca's total latency, on average, to perform hyperparameter
%tuning.  On average, the Seneca memory optimization increases latency by X\%
%on average across applications.  \textcolor{red}{Compute and fill in X here.}
%\textcolor{blue}{Add analysis here to discuss why the performance difference
%is different for different applications (some more sensitive to memory
%restrictions and why), and anything else you can pull out that is interesting
%in these numbers.}



In this section, we overview related work. We discuss contributions that focus
on serverless computing and machine learning model optimization. 

Multiple researchers have investigated the efficacy and overhead of the
serverless programming model and implementations.
the benefit and drawback of such has been well researched by
The authors in \cite{ref:onesteptwostep} evaluates the cost and performance of using
serverless for data-intensive computing.  They identify some key weaknesses in AWS
Lambda for this application domain including the inability to effectively train models.
Our work show that it is possible for a subset of machine learning applications to 
train models in a cost effective ways.
\cite{ref:baldini2017} identifies multiple open research questions surrounding serverless,
including whether or not functions should be stateless. PyWren~\cite{ref:jonas2017occupy} provides us 
a new perspective of stateless function being a natural fit for distributed computing models, because it 
abstract away cluster management overhead from running embarrassingly parallel jobs. ExCamera~\cite{ref:encoding}
 exemplifies a framework running general-purpose parallel tasks (encoding 4K video) on commercial serverless platform 
by multithreading. An empirical measurement study has been conducted in~\cite{ref:peeking} across AWS Lambda, 
Azure Functions and Google Cloud Functions to discuss the resource utilization and performance isolation efficiency of 
three platforms. Besides efficacy and efficiency, there are challenges serverless architecture are facing in terms of 
application monitoring and distributed storage. GammaRay~\cite{ref:lin2018tracking} proposes a cloud service that extracts
causal dependencies across serverless functions and cloud services without programmer intervention. Such monitoring and 
debugging tool simplifies the development and maintenance of serverless applications. Pocket~\cite{ref:ephemeral} provides 
an elastic and economical distributed data store for severless analytic applications. We expect more contributions to toolchain 
and distributed storage of serverless in upcoming years.

%please also discuss and cite these papers - you may need a second paragraph
%http://www.cs.ucsb.edu/~ckrintz/papers/ic2e18.pdf
%http://shivaram.org/publications/pywren-socc17.pdf
%S. Fouladi, R. S. Wahby, B. Shacklett, K. V. Balasubramaniam, W. Zeng, R. Bhalerao, A. Sivaraman, G. Porter, and K., Winstein. “Encoding, Fast and Slow: Low-Latency Video Pro- cessing Using Thousands of Tiny Threads”. In: NSDI. Boston, MA: USENIX Association, 2017, pp. 363–376.
%A. Klimovic, Y. Wang, C. Kozyrakis, P. Stuedi, J. Pfefferle, and A. Trivedi. “Understanding Ephemeral Storage for Serverless Analytics”. In: USENIX ATC. Boston, MA: USENIX Associa- tion, 2018, pp. 789–794.
%. Wang, M. Li, Y. Zhang, T. Ristenpart, and M. Swift. “Peek- ing Behind the Curtains of Serverless Platforms”. In: USENIX ATC. Boston, MA: USENIX Association, 2018, pp. 133–146.

The serverless framework~\cite{ref:serverless_framework} contributes an 
automated packaging and deployment toolset for serverless functions across clouds. 
The serverless framework uses Cloudformation~\cite{ref:cloudformation} to package and deploy Lambda function, which introduces additional cost. Seneca uses local docker container to avoid this cost
and overhead. The cloud infrastructure provisioning framework Terraform~\cite{ref:terraform} 
also provides automated deployment of functions to serverless platforms, but it does not simulate the AWS Lamdba runtime to create the deployment package (doing so on a non-Linux system will result in Lambda errors and failures at execution time). 
In addition, AWS CodeDeploy~\cite{ref:codedeploy} serves as component in the continuous integration and delivery pipeline for serverless architectures, but it also introduces monetary cost overhead
to do so.

Automated hyperparameter optimization has been researched and developed 
in numerous projects. Google Vizier~\cite{ref:vizier} provides a service for 
black-box optimization. Optunity~\cite{ref:claesen2014hyperparameter} 
proposes a Python library for hyperparameter tuning. 
This can be considered as precursory work of Hyperopt~\cite{ref:hyperopt}, 
which is a more advanced Python library for serial and parallel hyperparameter 
optimization over awkward search space. Hyperas ~\cite{ref:hyperas} adds another 
abstraction layer to hyperopt and facilitates the hyperparameter tuning for 
Keras ~\cite{ref:keras}. However, no extant work of which we are aware leverages
serverless to perform hyperparameter tuning in parallel. 

As related work, we consider recent advances
in evaluating serverless computing for different application domains, 
automatic deployment for serverless, and machine learning (ML) model optimization. 
For the former, much work has investigated the efficacy and overhead of the
serverless programming model and 
implementations~\cite{ref:jonas2017occupy,ref:onesteptwostep,ref:baldini2017,ref:lin2018tracking}.
%The authors in \cite{ref:onesteptwostep} evaluate the cost and performance of using
%serverless for data-intensive computing.  
The authors identify challenges with 
using AWS Lambda to train machine learning (ML) models.
Our work however, shows that it is possible to leverage the concurrency 
and parallelism in AWS Lambda to perform fast grid search for the
subset of ML applications that we consider.

PyWren~\cite{ref:jonas2017occupy} uses serverless for
different distributed computing models. The technique abstracts away cluster management overhead and is ideal for embarrassingly parallel jobs.
ExCamera~\cite{ref:encoding} presents a framework for running general-purpose parallel tasks (encoding 4K video) on a commercial serverless platform using multithreading. Cirrus~\cite{ref:jonas2019cloud} attempts to train ML models using a parameter server and serverless functions. 
%An empirical comparison of resource utilization and performance isolation
%is provided in~\cite{ref:peeking} for AWS Lambda, 
%Azure Functions, and Google Cloud Functions.

The serverless framework~\cite{ref:serverless_framework} provides
automated packaging and deployment for serverless functions across clouds. 
GammaRay~\cite{gammaray18} does so for AWS Lambda to insert profiling instrumentation.
The serverless framework uses 
CloudFormation~\cite{ref:cloudformation} for deployment in AWS Lambda, which introduces additional cost. 
The cloud infrastructure provisioning 
framework Terraform~\cite{ref:terraform}
also provides automated deployment of functions to serverless platforms.
%(however, without replicating the Lambda runtime, introducing deployment errors).
Seneca uses a local docker container to avoid  cost and overhead (vs these related
works), which guarantees execution compatibility for AWS Lambda. 
%In addition, AWS CodeDeploy~\cite{ref:codedeploy} serves as component in the continuous integration and delivery pipeline for serverless architectures, but it also introduces monetary cost overhead to do so.

Automated hyperparameter tuning is the focus of
many projects. Google Vizier~\cite{ref:vizier} provides a service for 
black-box optimization. Optunity~\cite{ref:claesen2014hyperparameter} 
and Hyperopt~\cite{ref:hyperopt}
provide a Python library for hyperparameter tuning. 
%This can be considered as precursory work of Hyperopt~\cite{ref:hyperopt}, 
%which is a more advanced Python library for serial and parallel hyperparameter 
%optimization over awkward search space. 
Hyperas~\cite{ref:hyperas} adds another 
abstraction layer to hyperopt to facilitate hyperparameter tuning for 
Keras~\cite{ref:keras}. However, no extant work of which 
we are aware leverages
serverless to perform hyperparameter tuning and memory optimization in parallel 
for ML applications.
